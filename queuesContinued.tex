\documentclass[]{article}
\usepackage{lmodern}
\usepackage{amssymb,amsmath}
\usepackage{ifxetex,ifluatex}
\usepackage{fixltx2e} % provides \textsubscript
\ifnum 0\ifxetex 1\fi\ifluatex 1\fi=0 % if pdftex
  \usepackage[T1]{fontenc}
  \usepackage[utf8]{inputenc}
\else % if luatex or xelatex
  \ifxetex
    \usepackage{mathspec}
  \else
    \usepackage{fontspec}
  \fi
  \defaultfontfeatures{Ligatures=TeX,Scale=MatchLowercase}
\fi
% use upquote if available, for straight quotes in verbatim environments
\IfFileExists{upquote.sty}{\usepackage{upquote}}{}
% use microtype if available
\IfFileExists{microtype.sty}{%
\usepackage{microtype}
\UseMicrotypeSet[protrusion]{basicmath} % disable protrusion for tt fonts
}{}
\usepackage[margin=1in]{geometry}
\usepackage{hyperref}
\hypersetup{unicode=true,
            pdftitle={Modeling, Simulating, and Measuring the Performance of Queues},
            pdfauthor={Anthony Hung},
            pdfborder={0 0 0},
            breaklinks=true}
\urlstyle{same}  % don't use monospace font for urls
\usepackage{color}
\usepackage{fancyvrb}
\newcommand{\VerbBar}{|}
\newcommand{\VERB}{\Verb[commandchars=\\\{\}]}
\DefineVerbatimEnvironment{Highlighting}{Verbatim}{commandchars=\\\{\}}
% Add ',fontsize=\small' for more characters per line
\usepackage{framed}
\definecolor{shadecolor}{RGB}{248,248,248}
\newenvironment{Shaded}{\begin{snugshade}}{\end{snugshade}}
\newcommand{\AlertTok}[1]{\textcolor[rgb]{0.94,0.16,0.16}{#1}}
\newcommand{\AnnotationTok}[1]{\textcolor[rgb]{0.56,0.35,0.01}{\textbf{\textit{#1}}}}
\newcommand{\AttributeTok}[1]{\textcolor[rgb]{0.77,0.63,0.00}{#1}}
\newcommand{\BaseNTok}[1]{\textcolor[rgb]{0.00,0.00,0.81}{#1}}
\newcommand{\BuiltInTok}[1]{#1}
\newcommand{\CharTok}[1]{\textcolor[rgb]{0.31,0.60,0.02}{#1}}
\newcommand{\CommentTok}[1]{\textcolor[rgb]{0.56,0.35,0.01}{\textit{#1}}}
\newcommand{\CommentVarTok}[1]{\textcolor[rgb]{0.56,0.35,0.01}{\textbf{\textit{#1}}}}
\newcommand{\ConstantTok}[1]{\textcolor[rgb]{0.00,0.00,0.00}{#1}}
\newcommand{\ControlFlowTok}[1]{\textcolor[rgb]{0.13,0.29,0.53}{\textbf{#1}}}
\newcommand{\DataTypeTok}[1]{\textcolor[rgb]{0.13,0.29,0.53}{#1}}
\newcommand{\DecValTok}[1]{\textcolor[rgb]{0.00,0.00,0.81}{#1}}
\newcommand{\DocumentationTok}[1]{\textcolor[rgb]{0.56,0.35,0.01}{\textbf{\textit{#1}}}}
\newcommand{\ErrorTok}[1]{\textcolor[rgb]{0.64,0.00,0.00}{\textbf{#1}}}
\newcommand{\ExtensionTok}[1]{#1}
\newcommand{\FloatTok}[1]{\textcolor[rgb]{0.00,0.00,0.81}{#1}}
\newcommand{\FunctionTok}[1]{\textcolor[rgb]{0.00,0.00,0.00}{#1}}
\newcommand{\ImportTok}[1]{#1}
\newcommand{\InformationTok}[1]{\textcolor[rgb]{0.56,0.35,0.01}{\textbf{\textit{#1}}}}
\newcommand{\KeywordTok}[1]{\textcolor[rgb]{0.13,0.29,0.53}{\textbf{#1}}}
\newcommand{\NormalTok}[1]{#1}
\newcommand{\OperatorTok}[1]{\textcolor[rgb]{0.81,0.36,0.00}{\textbf{#1}}}
\newcommand{\OtherTok}[1]{\textcolor[rgb]{0.56,0.35,0.01}{#1}}
\newcommand{\PreprocessorTok}[1]{\textcolor[rgb]{0.56,0.35,0.01}{\textit{#1}}}
\newcommand{\RegionMarkerTok}[1]{#1}
\newcommand{\SpecialCharTok}[1]{\textcolor[rgb]{0.00,0.00,0.00}{#1}}
\newcommand{\SpecialStringTok}[1]{\textcolor[rgb]{0.31,0.60,0.02}{#1}}
\newcommand{\StringTok}[1]{\textcolor[rgb]{0.31,0.60,0.02}{#1}}
\newcommand{\VariableTok}[1]{\textcolor[rgb]{0.00,0.00,0.00}{#1}}
\newcommand{\VerbatimStringTok}[1]{\textcolor[rgb]{0.31,0.60,0.02}{#1}}
\newcommand{\WarningTok}[1]{\textcolor[rgb]{0.56,0.35,0.01}{\textbf{\textit{#1}}}}
\usepackage{graphicx,grffile}
\makeatletter
\def\maxwidth{\ifdim\Gin@nat@width>\linewidth\linewidth\else\Gin@nat@width\fi}
\def\maxheight{\ifdim\Gin@nat@height>\textheight\textheight\else\Gin@nat@height\fi}
\makeatother
% Scale images if necessary, so that they will not overflow the page
% margins by default, and it is still possible to overwrite the defaults
% using explicit options in \includegraphics[width, height, ...]{}
\setkeys{Gin}{width=\maxwidth,height=\maxheight,keepaspectratio}
\IfFileExists{parskip.sty}{%
\usepackage{parskip}
}{% else
\setlength{\parindent}{0pt}
\setlength{\parskip}{6pt plus 2pt minus 1pt}
}
\setlength{\emergencystretch}{3em}  % prevent overfull lines
\providecommand{\tightlist}{%
  \setlength{\itemsep}{0pt}\setlength{\parskip}{0pt}}
\setcounter{secnumdepth}{0}
% Redefines (sub)paragraphs to behave more like sections
\ifx\paragraph\undefined\else
\let\oldparagraph\paragraph
\renewcommand{\paragraph}[1]{\oldparagraph{#1}\mbox{}}
\fi
\ifx\subparagraph\undefined\else
\let\oldsubparagraph\subparagraph
\renewcommand{\subparagraph}[1]{\oldsubparagraph{#1}\mbox{}}
\fi

%%% Use protect on footnotes to avoid problems with footnotes in titles
\let\rmarkdownfootnote\footnote%
\def\footnote{\protect\rmarkdownfootnote}

%%% Change title format to be more compact
\usepackage{titling}

% Create subtitle command for use in maketitle
\newcommand{\subtitle}[1]{
  \posttitle{
    \begin{center}\large#1\end{center}
    }
}

\setlength{\droptitle}{-2em}

  \title{Modeling, Simulating, and Measuring the Performance of Queues}
    \pretitle{\vspace{\droptitle}\centering\huge}
  \posttitle{\par}
    \author{Anthony Hung}
    \preauthor{\centering\large\emph}
  \postauthor{\par}
      \predate{\centering\large\emph}
  \postdate{\par}
    \date{2019-03-01}

\usepackage{outlines}
\usepackage{enumitem}
\usepackage{tikz}
\usetikzlibrary{arrows}
\usepackage{tabularx}

\begin{document}
\maketitle

\#Prerequisites

This vignette continues from concepts covered in the vignette:
``Introduction to Queueing Theory and Queueing Models''.

\#Introduction

In addition to simply being able to represent waiting lines
mathematically, queueing theory allows for the evaluation of the
behavior and performance of queues. Being able to measure the
performance of queues also allows us to determine the effects of
altering components of the queue on performance.

\#Simulating a M/M/1 queue

In simulating a M/M/1 queue, we want to keep track of three values of
the queue over time.

\begin{outline}[enumerate]
   \1 Arrival times of customers
   \1 Departure times of customers
   \1 The number of customers in the system at every moment of arrivals or departures
\end{outline}

In simulating the queue behavior, we can take advantage of the
superposition property of combined independent Poisson processes. Since
arrivals and departures are indepdendent, the number of events in the
combined process can be represented as a Poisson process with parameter
\(\lambda_{sum} = \lambda + \mu\). The probability of an event in this
combined process being an arrival is \(\frac{\lambda}{(\lambda+\mu)}\),
and the probabilty of it being a departure is
\(\frac{\mu}{(\lambda+\mu)}\).

The function ``simulate\_MM1'' simulates the number of customers in a
M/M/1 queue over time given values for lambda, mu, and \(N_0\) from
\(T_0\) to \(T_{max}\). It also keeps track of when events (arrivals or
departures) occur during the time periods and what type of event occurs
at each of those moments.

\begin{Shaded}
\begin{Highlighting}[]
\NormalTok{lambda <-}\StringTok{ }\DecValTok{4}
\NormalTok{mu <-}\StringTok{ }\DecValTok{5}

\NormalTok{simulate_MM1 <-}\StringTok{ }\ControlFlowTok{function}\NormalTok{(}\DataTypeTok{lambda=}\NormalTok{lambda, }\DataTypeTok{mu=}\NormalTok{mu, }\DataTypeTok{N0=}\DecValTok{0}\NormalTok{, }\DataTypeTok{Tmax=}\DecValTok{1000}\NormalTok{)\{}
  \CommentTok{#Initialize vectors to store each of the values of interest throughout the simulation}
\NormalTok{  events <-}\StringTok{ }\DecValTok{0} \CommentTok{#stores the type of event (1 for arrival, -1 for departure)}
\NormalTok{  Times <-}\StringTok{ }\DecValTok{0} \CommentTok{#times of events}
\NormalTok{  customers <-}\StringTok{ }\NormalTok{N0 }\CommentTok{#number of customers at each time in Times}
  
  \ControlFlowTok{while}\NormalTok{(}\KeywordTok{tail}\NormalTok{(Times,}\DecValTok{1}\NormalTok{) }\OperatorTok{<}\StringTok{ }\NormalTok{Tmax)\{ }\CommentTok{#keep simulating until you have an event at a time greater than Tmax}
    
    \ControlFlowTok{if}\NormalTok{(}\KeywordTok{tail}\NormalTok{(customers,}\DecValTok{1}\NormalTok{)}\OperatorTok{==}\DecValTok{0}\NormalTok{)\{ }\CommentTok{#separate behavior occurs if system currently has 0 customers}
\NormalTok{      tau <-}\StringTok{ }\KeywordTok{rexp}\NormalTok{(}\DecValTok{1}\NormalTok{, }\DataTypeTok{rate=}\NormalTok{lambda) }\CommentTok{#interarrival intervals are exponentially distributed}
\NormalTok{      event <-}\StringTok{ }\DecValTok{1} \CommentTok{#only an arrival can occur if thre are 0 customers}
      
\NormalTok{    \} }\ControlFlowTok{else}\NormalTok{ \{}
\NormalTok{      tau <-}\StringTok{ }\KeywordTok{rexp}\NormalTok{(}\DecValTok{1}\NormalTok{, }\DataTypeTok{rate=}\NormalTok{lambda}\OperatorTok{+}\NormalTok{mu) }\CommentTok{#inter-event intervals are exponentially distributed}
      \ControlFlowTok{if}\NormalTok{(}\KeywordTok{runif}\NormalTok{(}\DecValTok{1}\NormalTok{,}\DecValTok{0}\NormalTok{,}\DecValTok{1}\NormalTok{) }\OperatorTok{<}\StringTok{ }\NormalTok{lambda}\OperatorTok{/}\NormalTok{(lambda}\OperatorTok{+}\NormalTok{mu))\{ }\CommentTok{#if runif is less than P(event = arrival)...}
\NormalTok{        event <-}\StringTok{ }\DecValTok{1} \CommentTok{#call the event an arrival}
\NormalTok{      \} }\ControlFlowTok{else}\NormalTok{\{ }
\NormalTok{        event <-}\StringTok{ }\DecValTok{-1} \CommentTok{#otherwise, call the event a departure}
\NormalTok{      \}}
\NormalTok{    \}}
    
    \CommentTok{#now that we have simulatd one event, we need to do some accounting}
\NormalTok{    customers <-}\StringTok{ }\KeywordTok{c}\NormalTok{(customers, }\KeywordTok{tail}\NormalTok{(customers,}\DecValTok{1}\NormalTok{)}\OperatorTok{+}\NormalTok{event)}
\NormalTok{    Times <-}\StringTok{ }\KeywordTok{c}\NormalTok{(Times, }\KeywordTok{tail}\NormalTok{(Times,}\DecValTok{1}\NormalTok{)}\OperatorTok{+}\NormalTok{tau)}
\NormalTok{    events <-}\StringTok{ }\KeywordTok{c}\NormalTok{(events, event)}
\NormalTok{  \}}
  
  \CommentTok{#we need to toss out the information from the last event (it occured after Tmax)}
\NormalTok{  events <-}\StringTok{ }\KeywordTok{head}\NormalTok{(events, }\DecValTok{-1}\NormalTok{)}
\NormalTok{  Times <-}\StringTok{ }\KeywordTok{head}\NormalTok{(Times, }\DecValTok{-1}\NormalTok{)}
\NormalTok{  customers <-}\StringTok{ }\KeywordTok{head}\NormalTok{(customers, }\DecValTok{-1}\NormalTok{)}
  
  \KeywordTok{return}\NormalTok{(}\KeywordTok{list}\NormalTok{(events,Times,customers))}
\NormalTok{\}}
\end{Highlighting}
\end{Shaded}

After simulating the number of customers in the queue over one run of
the simulation, we can plot it.

\begin{Shaded}
\begin{Highlighting}[]
\NormalTok{sim <-}\StringTok{ }\KeywordTok{simulate_MM1}\NormalTok{(}\DataTypeTok{lambda =} \DecValTok{4}\NormalTok{, }\DataTypeTok{mu=}\DecValTok{5}\NormalTok{, }\DataTypeTok{N0=}\DecValTok{0}\NormalTok{, }\DataTypeTok{Tmax=}\DecValTok{1000}\NormalTok{)}
\KeywordTok{plot}\NormalTok{(}\DataTypeTok{x=}\NormalTok{sim[[}\DecValTok{2}\NormalTok{]], }\DataTypeTok{y=}\NormalTok{sim[[}\DecValTok{3}\NormalTok{]], }\DataTypeTok{xlab=}\StringTok{"time"}\NormalTok{, }\DataTypeTok{ylab=}\StringTok{"Number of customers"}\NormalTok{)}\CommentTok{#number of customers vs time}
\end{Highlighting}
\end{Shaded}

\includegraphics{queuesContinued_files/figure-latex/unnamed-chunk-1-1.pdf}

Notice that if \(\lambda > \mu\), the customer number explodes and will
never reach a steady state.

\begin{Shaded}
\begin{Highlighting}[]
\NormalTok{sim2 <-}\StringTok{ }\KeywordTok{simulate_MM1}\NormalTok{(}\DataTypeTok{lambda =} \DecValTok{6}\NormalTok{, }\DataTypeTok{mu=}\DecValTok{5}\NormalTok{, }\DataTypeTok{N0=}\DecValTok{0}\NormalTok{, }\DataTypeTok{Tmax=}\DecValTok{1000}\NormalTok{)}
\KeywordTok{plot}\NormalTok{(}\DataTypeTok{x=}\NormalTok{sim2[[}\DecValTok{2}\NormalTok{]], }\DataTypeTok{y=}\NormalTok{sim2[[}\DecValTok{3}\NormalTok{]], }\DataTypeTok{xlab=}\StringTok{"time"}\NormalTok{, }\DataTypeTok{ylab=}\StringTok{"Number of customers"}\NormalTok{)}\CommentTok{#number of customers vs time}
\end{Highlighting}
\end{Shaded}

\includegraphics{queuesContinued_files/figure-latex/unnamed-chunk-2-1.pdf}

\#Measuring the performance of queues

There are several formal quantities used to measure the performance of a
queueing system (with c servers).

\begin{outline}[enumerate]
   \1 $\pi_j$ := The stationary probability that there are j customers in the system
   \1 $a$ := Offered load. The mean number of requests per service time.
   \1 $\rho$ := Traffic intensity. Offered load per server ($a/c$).
   \1 $a'$ := Carried load. Mean number of busy servers.
   \1 $\rho'$ := Server occupancy. Carried load per server ($a'/c$).
   \1 $W_s$ := Mean length of time between a customer's arrival and the customer's departure from the system.
   \1 $W_q$ := Mean length of time between a customer's arrival and when the customer's service starts.
   \1 $L_s$ := Mean number of customers in the system, including those in the buffer and at servers.
   \1 $L_q$ := Mean number of customers waiting in the buffer.
\end{outline}

We can analyze the performance of a M/M/1 queue.

\underline{$\pi_j$}

For the M/M/1 queue, we previously calculated the stationary
probabilities:

\[\pi_n = \frac{\lambda^n}{\mu^n}\frac{1}{1+\sum\limits_{a=1}^{\infty} \frac{\lambda ^a}{\mu ^a}}\]

\underline{$a$, $\rho$}

The offered load of a server is given by the ratio of the arrival rate
to the departure rate. Traffic intensity is the offered load per server.

\[a = \frac{\lambda}{\mu}\]
\[\rho = \frac{a}{c} = \frac{\frac{\lambda}{\mu}}{1} = \frac{\lambda}{\mu}\]

\underline{$a'$}

\(a'\) and \(a\) are both dimensionless and expressed in ``erlang''
units.

\underline{A brief interlude: Little's Law}

Little's law states that the long-term average of the number of
customers in any queue at stationarity is equal to the long-term average
arrival rate λ multiplied by the average time that a customer spends in
the system. Expressed algebraically using our defined variables:

\[L_s = \lambda W_s\]

Little's law also holds for the number of customers waiting in the queue
buffer:

\[L_q = \lambda W_q\]

Little's Law was originally presented by John Little in 1954 without
proof, but multiple proofs of the relationship have been published since
(\url{https://pubsonline.informs.org/doi/abs/10.1287/opre.20.6.1115}).

\#Multiple servers: the M/M/c queue


\end{document}
